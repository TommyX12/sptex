\documentclass[11pt]{article}


\usepackage{amsmath}
\usepackage{lipsum}
    
    
    
\begin{document}
    
        \noindent
        A demo of put(): \\
        Given that the average customers per hour is 4, \\
        The probability that 1 customer comes within the next hour is 0.073263. \\
        The probability that 2 customers comes within the next hour is 0.146525. \\
        The probability that 3 customers comes within the next hour is 0.195367. \\
        The probability that 4 customers comes within the next hour is 0.195367. \\
        
        \noindent
        A second paragraph also without indent. \\
    
    Lists:
    \begin{itemize}
        \item  2 + 2 = 4
        \item  3 + 3 = 6
        \item  15 choose 4 is: 1365

        \item  22 choose 7 is: 170544
    \end{itemize}
    
    \begin{enumerate}
        \item  \lipsum[1]
            Embedded equation:
            \begin{align*}\zeta(s) &= \sum_{n = 1}^{\infty} \frac{1}{{n^s}} = \lim_{N \to \infty} \left(\sum_{n = 1}^{N} \frac{1}{{n^s}}\right)
            \end{align*}
            
        \item  \lipsum[2]
        \item  Nested enumerations
        \begin{enumerate}% short form, similar to calling functions
            \item  \lipsum[3]
            \item  \lipsum[4]
        \end{enumerate}
    \end{enumerate}
    
    \begin{align*}
        P(0 \leq S \leq \frac{5}{2}) &= \int_{0}^{\frac{5}{2}} \frac{1}{2} \cdot \frac{{4s^2}}{625} ds = 5 \\
        SP &= PS \\
        1 + \frac{1}{2}+\frac{1}{4}+\frac{1}{8}+\frac{1}{16} + \hdots &= 2 \\
        \frac{1}{2} \cdot \frac{1}{2} \cdot \hdots &= 0 = \frac{1}{\infty} \\
        \sum_{i = 1}^{5} i^2 &= 1^2+2^2+3^2+4^2+5^2 \\
        &= 1+4+9+16+25 = 55 \\
        \prod_{i = 1}^{5} i &= 5! = 1 \cdot 2 \cdot 3 \cdot 4 \cdot 5 = 120 \\
        \frac{1}{2} + \frac{1}{3} + \frac{1}{4} &\approx 1.08333
    \end{align*}
    
    \begin{center}\begin{tabular}{|c||l|l|l|}
        \hline 
        1 & 2 & 3 & 4 \\ \hline 
        5 & 6 & 7 & 8 \\ \hline 
        1 & 2 & 3 & 4 \\ \hline 
        5 & 6 & 7 & 8 \\ \hline 
    \end{tabular}
    \end{center}
    
    {\ttfamily 
        def f(n): \\
            if n <= 0: \\
                return 1 \\
            
            return n * f(n - 1) \\}
\end{document}